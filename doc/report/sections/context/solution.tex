\section{Solution}

To address the constraints described in the previous section, the original application-specific
problem was abstracted into a more general and formal model. In particular, the swarm of drones
is represented as a \textbf{complete directed graph}, where each node corresponds to a drone and
each directed edge represents the possibility of transmitting a message from one node to another.

The graph is assumed to be complete in order to model the broadcast nature of the underlying radio
medium: every node may transmit messages that can potentially be received by all other nodes.
Directionality is preserved to allow asymmetric communication behavior, such as message loss or
reception failure at specific nodes.

This abstraction enables several extensions without modifying the core protocol, including but
not limited to:
\begin{itemize}
  \item \textbf{Message loss modeling}: node-specific or edge-specific weights can be used to
    represent probabilistic message loss during reception.
  \item \textbf{Partial reachability}: removing edges from the graph allows modeling
    non-complete topologies, which can be used to study routing extensions or network partitioning.
\end{itemize}

An additional advantage of this formulation is that it allows the protocol to be evaluated in a
general-purpose computing environment. The graph-based model can be directly implemented in a
simulation running on a standard operating system, enabling rapid development, debugging, and
evaluation before deployment on physical hardware. This approach also allows experiments to be
conducted with a significantly larger number of nodes than those available in the real system.

To support both simulation and deployment, an abstract logging interface was introduced. During
simulation, where memory and computational constraints are relaxed, detailed execution traces can
be collected and stored. Most of the empirical data presented in this work is derived from logs
produced in the simulation environment, rather than from on-device execution.


% \section{Solution}
%
% To face all the mentioned issued I reduced the specific problem to a bigger and more general 
% problem. In particular the storm of drones has been reduced with: \textbf
% {
%   A complete directional graph
% }
% where each node is a drone and every arch indicates that the node a message can be sent from a 
% source to a destination.
% The graph is complete to indicate that every node can communicate with each other, both sending 
% and receiving messages.
%
% Generally speaking using this type of models allow for all sort of expansions including, but not
% limiting:
% \begin{itemize}
%   \item percentage of message lost per node: using weight in the node to indicate the percentage of
%   failures in reading
%   \item Node reachability: if an arch is missing (non-complete graph) than a node is not directly
%   reachable. (Useful to define and implement a routing extension of the protocol)
% \end{itemize}
%
% Another big advantage is that a graph problem can be easily modeled in a general program on an OS 
% to create a \textbf{simulation} and only after that make a porting for the drones. Removing
% in this way the need to debug on the physical drones and still leaving the possibility the 
% test the protocol with a much greater number of \textbf{nodes}. Giving at the end better data, 
% faster development and multi-target deployment for different systems even outside the specific
% context of the drones.
%
% In the end an abstract logger has been added to simplify the development giving also the
% possibility to store logs for future analyze.
%
% In fact most of the data used in this paper comes from the logger obtained in the simulations
% where the constraint in memory and clock speed are not a problem.
