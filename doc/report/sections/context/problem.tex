\section{Context}

Before describing the behavior of the protocol, it is necessary to clarify the problem being
addressed and the constraints under which the solution operates.

The target application consists of a group of autonomous drones that must collectively determine
which node should act as a leader. Leader selection is based on locally measurable properties,
such as the quality of the wireless signal to an access point and the remaining battery level.
The selected leader is intended to act as the reference node for the swarm.

The physical platform used in this work is the \textbf{Crazyflie~2.1} quadcopter~\cite{1}.
While each drone can independently connect to a wireless access point, inter-drone communication
is performed exclusively through a shared radio channel operating in the 2.4\,GHz ISM band.
This communication medium provides no guarantees in terms of delivery, ordering, or Quality of
Service, and supports only broadcast transmission.

The available radio drivers do not allow addressing individual nodes directly, except through a
token-ring mechanism that proved too fragile for the intended use case. As a consequence, all
inter-node communication must be performed through unreliable broadcast messages delivered to all
reachable nodes.

These constraints motivated the design of a protocol that does not rely on any connection-oriented
communication mechanism, such as TCP~\cite{2} or QUIC~\cite{3}. While it would be theoretically
possible to port a connection-oriented protocol to the platform using custom drivers, the
additional control traffic required to maintain connections was deemed unsuitable for a dense and
interference-prone wireless environment. Instead, the protocol was designed to minimize
communication overhead and tolerate high message loss by construction.

Finally, it is important to note that the number of physical drones available for experimentation
is limited to a maximum of four units. As a result, large-scale behavior cannot be validated
directly on hardware. To address this limitation, the protocol was designed to be
platform-independent and was evaluated through simulation in a Linux-based environment,
allowing experiments with a significantly larger number of nodes.


% \section{Context}
%
% Before explaining how the protocol behaves it is better to spend a few words explaining what is the
% problem I'm trying to solve and context in which it has to be solved.
%
% I have been assigned the following problem: \textbf
% {
%   given a bunch o drones connected to an access
%   point determine the best one based on the quality of the signal and the power level of the
%   battery.
% }
% I want to highlight an aspect of the problem, the drones are real and in particular they are the
% \textbf
% {
%   Crazyflie 2.1
% }
% \cite{1}. They can connect to a wireless access point but to communicate
% between them they use a radio (2.4GHz ISM band radio) connection with no \textbf
% {
%   Quality Of Service
% }.
% Since the aim of the problem was to design a protocol which would allow the drones to determine, 
% autonomously, which of them is the best one to be considered as the loader of the storm,
% I had to use the radio medium to exchange information between them.
%
% The drivers of the drones do not offer way to send a message to a specific drone, outside a
% token ring which is too fragile for my use case. The only feature they offer is sending a message
% in  broadcast to all the reachable nodes. 
%
% This is why I designed the protocol in such a way that I does not rely on any connection oriented
% protocol like \textbf{TCP}\cite{2} or \textbf{QUIC}\cite{3}. Also, I could port \textbf{TCP}\cite{2}
% in the drones with a custom driver, but I was scared that the overhead of the messages for the
% connection would cause an excessive amount of interference between the drones. On the other hand
% by designing the all stack to reduce the number of messages sent by each drone the problem was much
% easier to manage and control.
%
% Last and not least the number of drones at my disposal is limited to a maximum of \textbf{4}.
% That's probably the biggest of all the problems since it means that if I develop something that
% works with these drones is not guaranteed to work with much greater numbers.
