\section{Useful Definitions}

Before describing the protocol in detail, several terms used throughout this work are formally
defined.

\begin{definition}[Potential]\label{def:potential}
  The \textbf{potential} of a node is a non-negative integer $p \in \mathbb{N}$ computed locally by
  each node. It represents an application-dependent measure of the node's suitability to act as
  leader (e.g., remaining battery level, signal quality, or a combination thereof).
  The objective of the protocol is to select a node with maximal potential.
\end{definition}

\begin{definition}[Tolerance]\label{def:tolerance}
  The \textbf{tolerance} is a non-negative integer $t \in \mathbb{N}$ used when comparing node
  potentials. A node is considered strictly better than another only if its potential exceeds the
  other's potential by more than $t$. The tolerance parameter is introduced to reduce oscillations
  and excessive leader re-elections. A tolerance value of zero disables this mechanism.
\end{definition}

\begin{definition}[Best Candidate]\label{def:best_candidate}
  The \textbf{best candidate} is the node currently known to have the highest potential among all
  observed nodes. This notion is based on locally available information and may differ across
  nodes due to message loss or delays.
\end{definition}

\begin{definition}[Heartbeat]\label{def:heartbeat}
  A \textbf{heartbeat} is a periodically refreshed counter used to infer the continued activity
  of a node, particularly the current leader. If the counter is not refreshed within a given time
  window, the corresponding node is assumed to be unreachable or inactive for the purposes of
  leader election.
\end{definition}

\begin{definition}[Leader Check]\label{def:leader_check}
  A node declares itself leader if the following conditions hold:
  \begin{itemize}
    \item the node has won the most recent election round;
    \item the node's potential is greater than or equal to the potential of the current best
      candidate.
  \end{itemize}
\end{definition}


% \section{Useful Definition}
% Before jumping to the solution let's define some useful definition that will be used from now on:
%
% \begin{definition}[Potential]\label{def:potential}
%   The potential is an integer $\in \N$ which is computed by each node and represent the current
%   power of a node. Ideally the leader is the node with the greatest potential.
% \end{definition}
%
% \begin{definition}[Tolerance]\label{def:tollerance}
%   The tolerance is an integral $\in \N$ that is used when comparing the current potential of a node
%   with the stored potential of the leader. It's added to prevent too many elections and give
%   stability to the system. It's possible that because of the tolerance the leader is not the best
%   node between all the possible node. The default value is 0.
% \end{definition}
%
% \begin{definition}[Best candidate]\label{def:best_candidate}
%   The \textbf{best candidate} in a graph is the node with the greatest
%   \textbf{potential}[\ref{def:potential}].
% \end{definition}
%
% \begin{definition}[Hearbit]\label{def:heartbit}
%   A counter $\in \N$ which is used to determine if a certain node is still alive.
%   If the $\textbf{heartbit} = 0$ the node is considered dead.
%   Else the node is considered alive.
% \end{definition}
%
% \begin{definition}[Leader check]\label{def:leader_check}
%   A node define himself as leader if the following conditions are met:
%   \begin{itemize}
%     \item node won the election
%     \item node potential >= best candidate[\ref{def:best_candidate}]
%       potential [\ref{def:potential}]
%   \end{itemize}
%   
% \end{definition}
%
