\section{Actual Solution}

The solution implemented in \textbf{NearestAP} is inspired by the leader election mechanism of
\textbf{RAFT}\cite{4}, but is adapted to operate over unreliable broadcast media without any
connection-oriented guarantees.

Each node maintains a local view of the network, including the current leader and the
\textbf{best candidate}(see Definition~\ref{def:best_candidate}). Nodes infer leadership
by observing the continuous dissemination of identifiers, potentials, and election-related messages,
rather than relying on direct acknowledgments.

To prevent persistent election conflicts and excessive message traffic, NearestAP introduces an
adaptive election timing mechanism. Election attempts are not performed at a fixed rate; instead,
each node dynamically adjusts the frequency of election initiation based on observed leader
stability, candidate dominance, and network silence. Nodes that are unlikely to win an election
progressively suppress their own election activity, while the most promising candidate retains a
higher election rate.

Adaptive timing is implemented through a local election time scale factor.
Nodes that repeatedly attempt elections without being the strongest known candidate progressively
reduce their election frequency, allowing dominant candidates to emerge without continuous
contention. Conversely, prolonged silence from both the leader and the best candidate causes
nodes to restore aggressive election behavior, ensuring recovery from message loss or network
partitions.

In the absence of both leader and best-candidate signals, nodes autonomously recover by resetting
their election timing, ensuring continued progress even under severe message loss.
