\section{Actual Solution}

The protocol implemented in \textbf{NearestAP} is inspired by classical leader election mechanisms,
including those employed by Raft~\cite{4}, but is specifically designed to operate under
unreliable broadcast communication and high message loss. Rather than replicating the full Raft
election procedure, the protocol adopts a lightweight, best-effort approach that minimizes
communication overhead and relaxes consistency guarantees.

The central idea of the protocol is to reduce the number of election attempts by allowing nodes
to infer the most likely leader based on information opportunistically gathered from the network.
Each node maintains a locally observed \textbf{best candidate}, defined as the node with the
highest known potential according to messages received so far.

To support this inference mechanism, every message exchanged in the system carries the identifier
and potential of a \emph{candidate} node.
In leader heartbeats and election messages, this candidate corresponds to the sender itself,
while in vote response messages it corresponds to the node being endorsed.

Upon receiving any message, a node
updates its local view of the best candidate and suppresses its own election attempts if it
becomes aware of another node with a strictly higher potential.

Leader recognition is therefore conditional and local: a node considers another node to be the
leader only if that node has won the most recent election round and is also the current best
candidate according to the node’s locally available information. This design allows leadership
information to propagate implicitly through the network, even in the presence of message loss,
without requiring reliable delivery or explicit acknowledgments.


% \section{Actual Solution}
% The real solution that I implemented in \textbf{Nearest AP} is a variation of the election
% protocol of \textbf{RAFT}\cite{4}.
% The protocol aim to minimize the message sent on the bus by making the nodes infer who should 
% probably be the leader.
% Each node that using the \textbf{best candidate} which is the node with the greatest potential
% in the node knowledge obtained by reading the messages in the network.
% Every time a node reads a message it checks if the sender is the \textbf{best candidate}
% or if it can be a new one. For this reason \textbf{every message} has an id of a node and the
% potential of that node.
% Using this extra data a node p recognize a node l as a leader only if l won the election and if 
% the \textbf{best candidate} for p.
