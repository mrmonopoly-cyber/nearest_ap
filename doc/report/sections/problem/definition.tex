\section{Problem Definition}

Based on the context introduced above, the problem addressed by this work can now be defined
formally.

\begin{problem}[Leader Election]\label{prob:leader_election}
  Let $G = (V, E)$ be a complete-directed graph, where each node $n \in V$ is associated with a
  scalar \textbf{potential} $p(n)$ (see Definition~\ref{def:potential}).
  Communication between nodes occurs by exchanging messages along directed edges. Message delivery is unreliable: each transmitted message may be lost with non-zero probability.

  The objective is to determine a \textbf{leader} node $n^\ast \in V$ such that
  \[
    n^\ast \in \arg\max_{n \in V} p(n).
  \]
\end{problem}

The problem must be solved in a continuous and dynamic manner, as node potentials may change
over time. Consequently, the protocol is required to adapt to variations in potentials and
re-elect a leader when necessary.


% \section{Problem Definition}
% With the above context now let's properly define what is the problem that has to be solved:
% \begin{problem}[Leader Election]
%   Let $G=(V,E)$ a complete directed graph where each \textbf{node} n has
%   \textbf{potential}[\ref{def:potential}] p
%   and each node can send messages to the other nodes through the arch.
%   Each message has a certain probability $k$ of not being received.
%   Determine the \textbf{leader} p as follows:
%   \begin{center}
%     leader node = $\max_{n \in V}(n.p >= m.p | m \in V)$
%   \end{center}
% \end{problem}
% It's important to point out that the protocol has to dynamically continuously solve the problem
% since the potentials[\ref{def:potential}] can change.
