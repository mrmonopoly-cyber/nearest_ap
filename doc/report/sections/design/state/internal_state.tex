\subsection{Node Internal State}

Each node maintains a local internal state composed of several logically distinct components:

\begin{itemize}
  \item \textbf{Topology}: describes the assumed network configuration, including the total number
    of nodes.
    \footnote{
      In the current version of the protocol, the network topology is static and cannot change at
      runtime.
      Nodes may become temporarily unreachable due to message loss, but they are still considered
      members of the network.
    }

  \item \textbf{Node Potentials} (Definition~\ref{def:potential}): the node stores the potential
    values of:
    \begin{enumerate}
      \item the local node;
      \item the current leader;
      \item the current best candidate (Definition~\ref{def:best_candidate}).
    \end{enumerate}

  \item \textbf{Heartbeat Counters}: two counters used to infer node activity:
    \begin{enumerate}
      \item a leader heartbeat counter, refreshed upon reception of valid leader heartbeat
        messages;
      \item a best candidate heartbeat counter, refreshed upon reception of messages referring to
        the current best candidate.
    \end{enumerate}

  \item \textbf{Node Identifiers}:
    \begin{enumerate}
      \item the identifier of the current leader;
      \item the identifier of the current best candidate.
    \end{enumerate}

  \item \textbf{Vote Information}: election-related state, including:
    \begin{enumerate}
      \item the total number of nodes in the network;
      \item the number of positive votes received during the current election;
      \item the current election round;
      \item two boolean flags indicating:
        \begin{itemize}
          \item whether the node has already initiated an election in the current round;
          \item whether the node has already voted in the current round.
        \end{itemize}
    \end{enumerate}
\end{itemize}


% \subsection{Node Internal State}
% Each node has an internal state which is divided in 4 categories:
% \begin{itemize}
%   \item Topology: tells the node the current network configuration
%     \footnote
%     {
%       At the current state of the protocol the network configuration is static and cannot alter.
%       Nodes can be unreachable, but they will still be considered as a valid member of the 
%       network.
%     }
%   \item Node Potentials[\ref{def:potential}]: contain the Potentials of:
%     \begin{enumerate}
%       \item current node
%       \item leader\label{state:leader}
%       \item best candidate[\ref{def:best_candidate}]
%     \end{enumerate}
%   \item \textbf{Leader heartbit} \label{def:laeder_heartbit}
%   \item \textbf{Best candidate[\label{def:best_candidate}] heartbit}\label{def:best_candidate_heartbit}
%   \item \textbf{Leader} id: the current leader node
%   \item \textbf{Best candidate}[\label{def:best_candidate}] id: the current best candidate node
%   \item Vote Info: the info regarding the elections. It contains:
%     \begin{enumerate}
%       \item the total number of nodes in the network\label{state:node_network}
%       \item the number of positive vote that the node has received\label{state:consent}
%       \item the current round\label{state:round}
%       \item two flags which indicates:
%         \begin{itemize}
%           \item if the node already sent an election in the current election\label{state:election_sent}
%           \item if the node already voted in the round current election\label{state:voted}
%         \end{itemize}
%         \label{state:flags}
%     \end{enumerate}
% \end{itemize}
