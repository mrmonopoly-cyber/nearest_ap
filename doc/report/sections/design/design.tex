\section{Protocol Design Overview}

The design of NearestAP follows a decentralized and event-driven approach. Each node operates
independently and maintains its own internal state, reacting to messages received from the network
and periodically executing local tasks.

The protocol is structured around three main components: internal state management, message
exchange, and scheduled tasks. Adaptive timing is treated as a first-class design concern, allowing
the protocol to balance responsiveness and stability under varying network conditions.

Communication between nodes occurs through a small set of message types, each carrying information
relevant to leader election, such as node identifiers, potentials, and election rounds. Messages
are exchanged using unreliable broadcast, and no assumptions are made regarding delivery
guarantees or ordering.

Protocol logic is executed through a set of periodic and reactive tasks. Periodic tasks are
responsible for broadcasting leader heartbeats and initiating elections when appropriate, while a
reactive task processes incoming messages and updates the internal state accordingly.
The interaction between these tasks allows nodes to infer leadership information and suppress
unnecessary elections.

The following subsections describe these components in detail, starting with the internal state
representation, followed by message formats and task behavior.


% \section{Design}
