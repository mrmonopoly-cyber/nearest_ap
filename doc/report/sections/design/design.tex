\section{Protocol Design Overview}

This section provides a high-level overview of the NearestAP protocol architecture.
The protocol is structured around three main elements: internal node state, message types,
and scheduled tasks. Together, these components implement a decentralized leader election
mechanism over unreliable broadcast communication.

Each node maintains a local internal state containing information about its own potential,
the currently known leader, the best candidate observed so far, and the status of any ongoing
election. This state is updated both periodically and in response to incoming messages.

Communication between nodes occurs through a small set of message types, each carrying information
relevant to leader election, such as node identifiers, potentials, and election rounds. Messages
are exchanged using unreliable broadcast, and no assumptions are made regarding delivery
guarantees or ordering.

Protocol logic is executed through a set of periodic and reactive tasks. Periodic tasks are
responsible for broadcasting leader heartbeats and initiating elections when appropriate, while a
reactive task processes incoming messages and updates the internal state accordingly.
The interaction between these tasks allows nodes to infer leadership information and suppress
unnecessary elections.

The following subsections describe these components in detail, starting with the internal state
representation, followed by message formats and task behavior.


% \section{Design}
