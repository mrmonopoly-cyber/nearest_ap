\subsection{Messages}

The protocol operates through a small set of message types exchanged using unreliable broadcast.
While the specific semantics of each message type are described in the following subsections,
all incoming messages are subject to a common set of preliminary checks.

Upon reception of any message, a node performs the following operations:

\begin{itemize}
  \item \textbf{Best candidate update}: the message is examined to determine whether it refers to
    a node that should be considered as the current best candidate
    (Definition~\ref{def:best_candidate}). In particular:
    \begin{enumerate}
      \item if the potential carried in the message is greater than the locally stored best
        candidate potential, the best candidate is updated accordingly;
      \item if the identifier carried in the message matches the current best candidate
        identifier, the corresponding heartbeat counter is refreshed.
    \end{enumerate}

  \item \textbf{Round consistency check}: if the message carries an election round number greater
    than or equal to the node's current round, the local round is updated if necessary.
    Messages referring to older rounds are ignored with respect to election state updates.
\end{itemize}

All message types include a logical \textbf{round} number $r \in \mathbb{N}$, which is used to
order election attempts and prevent obsolete information from influencing the current election
state.

In addition to updating logical state, the reception of heartbeats directly influences the
election timing mechanism by reducing the local election time scale factor.


% \subsection{Messages}
% The protocol work around the following message that each node can send.
% The following checks are executed for each message:
% \begin{itemize}
%   \item check if the message comes from the best candidate[\ref{def:best_candidate}]
%     in the network by checking if:
%     \begin{enumerate}
%       \item \textbf{potential}[\ref{def:potential}] is >= than
%         \textbf{current best candidate potential}[\ref{def:potential}].
%       \item id == current \textbf{best candidate}[\ref{def:best_candidate}] id.
%     \end{enumerate}
%     In both the cases the infos of the \textbf{best candidate} are updated and the 
%     \textbf{heartbit}[\ref{def:heartbit}] for the \textbf{best candidate}[\ref{def:best_candidate}] is increased.
%   \item Check if the \textbf{round} is greater or equal than its round.
%     If the message round < its round the heartbit[\ref{def:heartbit}] is discarded Else 
%     the node update its own \textbf{round.}
% \end{itemize}
% Each of them uses the \textbf{round} $\in \N$ to measure time.
