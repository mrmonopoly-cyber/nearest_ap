\documentclass[sigconf,natbib=true]{acmart}

\AtBeginDocument{%
  \providecommand\BibTeX{{%
    Bib\TeX}}}

%%
%% These commands are for a JOURNAL article.

%\citestyle{acmauthoryear}
\usepackage[english]{babel}
\usepackage[T1]{fontenc}
\usepackage{amsmath}
\usepackage{mathtools}
\usepackage{stmaryrd}
\usepackage{listings}
\usepackage{url}
\usepackage{amsfonts} % for "\mathbb" macro

\newcommand{\N}{\mathbb{N}}
\newcommand{\Z}{\mathbb{Z}}
\newcommand{\Q}{\mathbb{Q}}
\newcommand{\R}{\mathbb{R}}
\newcommand{\C}{\mathbb{C}}

\newtheorem{problem}{Problem}[section]

\newtheorem{definition}{Definition}[section]


\setcitestyle{numbers}

\title{NearestAP: a Leader Convergence Protocol for Broadcast-Only Clusters}
%%%%%%%%%%%%%%%%%%%%%%%%%%%%%%%%%%%%%%%%%%%%%%%%%%%%%%%%%%%%%%%%%%%%%%%%%%%%%%%%%%%%%%%%%%%%%%%%%%%%%%%%%%%%%%%%%%%%%%%%%%%%%%
%%%%%%%%%%%%%%%%%%%%%%%%%%%%%%%%%%%%%%%%%%%%%%%%%%%%%%%%%%%%%%%%%%%%%%%%%%%%%%%%%%%%%%%%%%%%%%%%%%%%%%%%%%%%%%%%%%%%%%%%%%%%%%
\author{Alberto Damo}
\email{alberto.damo@studenti.unipd.it}
\affiliation{%
  \institution{Università degli Studi di Padova}
  \city{Padova}
  \state{Veneto}
  \country{Italia}
}

%%
%% By default, the full list of authors will be used in the page
%% headers. Often, this list is too long, and will overlap
%% other information printed in the page headers. This command allows
%% the author to define a more concise list
%% of authors' names for this purpose.
\renewcommand{\shortauthors}{Damo}

%%
%% The abstract is a short summary of the work to be presented in the
%% article.
\begin{abstract}
This work presents NearestAP, a distributed protocol for eventual leader convergence in clusters
composed of autonomous nodes communicating exclusively through unreliable broadcast channels.
The protocol does not rely on any connection-oriented mechanism at any layer, making it suitable
for embedded and cyber-physical systems operating over non-conventional or lossy buses.

NearestAP enables a group of nodes to converge toward the selection of a single dominant leader
based on a locally defined potential metric. The protocol is resilient to extreme message loss
and does not assume synchronized clocks, reliable delivery, or bidirectional links. Experimental
evaluation through a Linux-based simulation framework shows that leader convergence is achieved
even when up to 90\% of messages are independently dropped at each node.

By favoring suppression and information diffusion over explicit coordination, NearestAP trades
optimal convergence time for robustness and simplicity, providing a practical solution for
leader selection in highly unreliable distributed environments.
\end{abstract}

% \begin{abstract}
%  This work presents NearestAP, a novel approach to determine the distributed consensus in a storm
%   of drones which do not rely on any form of connection-oriented protocol at any level. As I will show 
%   the protocol is able to eventually successfully determine the best drone in a group even if 90\% of the 
%   messages are lost for each drone. Since the protocol does not rely on any form of 
%   connection-oriented protocol it's particularly suited for the embedded world with non-conventional
%   bus.
% \end{abstract}


%%%%%%%%%%%%%%%%%%%%%%%%%%%%%%%%%%%%%%%%%%%%%%%%%%%%%%%%%%%%%%%%%%%%%%%%%%%%%%%%%%%%%%%%%%%%%%%%%%%%%%%%%%%%%%%%%%%%%%%%%%%%%%
%%%%%%%%%%%%%%%%%%%%%%%%%%%%%%%%%%%%%%%%%%%%%%%%%%%%%%%%%%%%%%%%%%%%%%%%%%%%%%%%%%%%%%%%%%%%%%%%%%%%%%%%%%%%%%%%%%%%%%%%%%%%%%
\begin{document}

\maketitle

\section{Context}

Before describing the behavior of the protocol, it is necessary to clarify the problem being
addressed and the constraints under which the solution operates.

The target application consists of a group of autonomous drones that must collectively determine
which node should act as a leader. Leader selection is based on locally measurable properties,
such as the quality of the wireless signal to an access point and the remaining battery level.
The selected leader is intended to act as the reference node for the swarm.

The physical platform used in this work is the \textbf{Crazyflie~2.1} quadcopter~\cite{1}.
While each drone can independently connect to a wireless access point, inter-drone communication
is performed exclusively through a shared radio channel operating in the 2.4\,GHz ISM band.
This communication medium provides no guarantees in terms of delivery, ordering, or Quality of
Service, and supports only broadcast transmission.

The available radio drivers do not allow addressing individual nodes directly, except through a
token-ring mechanism that proved too fragile for the intended use case. As a consequence, all
inter-node communication must be performed through unreliable broadcast messages delivered to all
reachable nodes.

These constraints motivated the design of a protocol that does not rely on any connection-oriented
communication mechanism, such as TCP~\cite{2} or QUIC~\cite{3}. While it would be theoretically
possible to port a connection-oriented protocol to the platform using custom drivers, the
additional control traffic required to maintain connections was deemed unsuitable for a dense and
interference-prone wireless environment. Instead, the protocol was designed to minimize
communication overhead and tolerate high message loss by construction.

Finally, it is important to note that the number of physical drones available for experimentation
is limited to a maximum of four units. As a result, large-scale behavior cannot be validated
directly on hardware. To address this limitation, the protocol was designed to be
platform-independent and was evaluated through simulation in a Linux-based environment,
allowing experiments with a significantly larger number of nodes.


% \section{Context}
%
% Before explaining how the protocol behaves it is better to spend a few words explaining what is the
% problem I'm trying to solve and context in which it has to be solved.
%
% I have been assigned the following problem: \textbf
% {
%   given a bunch o drones connected to an access
%   point determine the best one based on the quality of the signal and the power level of the
%   battery.
% }
% I want to highlight an aspect of the problem, the drones are real and in particular they are the
% \textbf
% {
%   Crazyflie 2.1
% }
% \cite{1}. They can connect to a wireless access point but to communicate
% between them they use a radio (2.4GHz ISM band radio) connection with no \textbf
% {
%   Quality Of Service
% }.
% Since the aim of the problem was to design a protocol which would allow the drones to determine, 
% autonomously, which of them is the best one to be considered as the loader of the storm,
% I had to use the radio medium to exchange information between them.
%
% The drivers of the drones do not offer way to send a message to a specific drone, outside a
% token ring which is too fragile for my use case. The only feature they offer is sending a message
% in  broadcast to all the reachable nodes. 
%
% This is why I designed the protocol in such a way that I does not rely on any connection oriented
% protocol like \textbf{TCP}\cite{2} or \textbf{QUIC}\cite{3}. Also, I could port \textbf{TCP}\cite{2}
% in the drones with a custom driver, but I was scared that the overhead of the messages for the
% connection would cause an excessive amount of interference between the drones. On the other hand
% by designing the all stack to reduce the number of messages sent by each drone the problem was much
% easier to manage and control.
%
% Last and not least the number of drones at my disposal is limited to a maximum of \textbf{4}.
% That's probably the biggest of all the problems since it means that if I develop something that
% works with these drones is not guaranteed to work with much greater numbers.

\section{Solution}

To address the constraints described in the previous section, the original application-specific
problem was abstracted into a more general and formal model. In particular, the swarm of drones
is represented as a \textbf{complete directed graph}, where each node corresponds to a drone and
each directed edge represents the possibility of transmitting a message from one node to another.

The graph is assumed to be complete in order to model the broadcast nature of the underlying radio
medium: every node may transmit messages that can potentially be received by all other nodes.
Directionality is preserved to allow asymmetric communication behavior, such as message loss or
reception failure at specific nodes.

This abstraction enables several extensions without modifying the core protocol, including but
not limited to:
\begin{itemize}
  \item \textbf{Message loss modeling}: node-specific or edge-specific weights can be used to
    represent probabilistic message loss during reception.
  \item \textbf{Partial reachability}: removing edges from the graph allows modeling
    non-complete topologies, which can be used to study routing extensions or network partitioning.
\end{itemize}

An additional advantage of this formulation is that it allows the protocol to be evaluated in a
general-purpose computing environment. The graph-based model can be directly implemented in a
simulation running on a standard operating system, enabling rapid development, debugging, and
evaluation before deployment on physical hardware. This approach also allows experiments to be
conducted with a significantly larger number of nodes than those available in the real system.

To support both simulation and deployment, an abstract logging interface was introduced. During
simulation, where memory and computational constraints are relaxed, detailed execution traces can
be collected and stored. Most of the empirical data presented in this work is derived from logs
produced in the simulation environment, rather than from on-device execution.


% \section{Solution}
%
% To face all the mentioned issued I reduced the specific problem to a bigger and more general 
% problem. In particular the storm of drones has been reduced with: \textbf
% {
%   A complete directional graph
% }
% where each node is a drone and every arch indicates that the node a message can be sent from a 
% source to a destination.
% The graph is complete to indicate that every node can communicate with each other, both sending 
% and receiving messages.
%
% Generally speaking using this type of models allow for all sort of expansions including, but not
% limiting:
% \begin{itemize}
%   \item percentage of message lost per node: using weight in the node to indicate the percentage of
%   failures in reading
%   \item Node reachability: if an arch is missing (non-complete graph) than a node is not directly
%   reachable. (Useful to define and implement a routing extension of the protocol)
% \end{itemize}
%
% Another big advantage is that a graph problem can be easily modeled in a general program on an OS 
% to create a \textbf{simulation} and only after that make a porting for the drones. Removing
% in this way the need to debug on the physical drones and still leaving the possibility the 
% test the protocol with a much greater number of \textbf{nodes}. Giving at the end better data, 
% faster development and multi-target deployment for different systems even outside the specific
% context of the drones.
%
% In the end an abstract logger has been added to simplify the development giving also the
% possibility to store logs for future analyze.
%
% In fact most of the data used in this paper comes from the logger obtained in the simulations
% where the constraint in memory and clock speed are not a problem.


\section{Problem Definition}

Based on the context introduced above, the problem addressed by this work can now be defined
formally.

\begin{problem}[Leader Election]\label{prob:leader_election}
  Let $G = (V, E)$ be a complete-directed graph, where each node $n \in V$ is associated with a
  scalar \textbf{potential} $p(n)$ (see Definition~\ref{def:potential}).
  Communication between nodes occurs by exchanging messages along directed edges. Message delivery is unreliable: each transmitted message may be lost with non-zero probability.

  The objective is to determine a \textbf{leader} node $n^\ast \in V$ such that
  \[
    n^\ast \in \arg\max_{n \in V} p(n).
  \]
\end{problem}

The problem must be solved in a continuous and dynamic manner, as node potentials may change
over time. Consequently, the protocol is required to adapt to variations in potentials and
re-elect a leader when necessary.


% \section{Problem Definition}
% With the above context now let's properly define what is the problem that has to be solved:
% \begin{problem}[Leader Election]
%   Let $G=(V,E)$ a complete directed graph where each \textbf{node} n has
%   \textbf{potential}[\ref{def:potential}] p
%   and each node can send messages to the other nodes through the arch.
%   Each message has a certain probability $k$ of not being received.
%   Determine the \textbf{leader} p as follows:
%   \begin{center}
%     leader node = $\max_{n \in V}(n.p >= m.p | m \in V)$
%   \end{center}
% \end{problem}
% It's important to point out that the protocol has to dynamically continuously solve the problem
% since the potentials[\ref{def:potential}] can change.

\section{Useful Definitions}

Before describing the protocol in detail, several terms used throughout this work are formally
defined.

\begin{definition}[Potential]\label{def:potential}
  The \textbf{potential} of a node is a non-negative integer $p \in \mathbb{N}$ computed locally by
  each node. It represents an application-dependent measure of the node's suitability to act as
  leader (e.g., remaining battery level, signal quality, or a combination thereof).
  The objective of the protocol is to select a node with maximal potential.
\end{definition}

\begin{definition}[Tolerance]\label{def:tolerance}
  The \textbf{tolerance} is a non-negative integer $t \in \mathbb{N}$ used when comparing node
  potentials. A node is considered strictly better than another only if its potential exceeds the
  other's potential by more than $t$. The tolerance parameter is introduced to reduce oscillations
  and excessive leader re-elections. A tolerance value of zero disables this mechanism.
\end{definition}

\begin{definition}[Best Candidate]\label{def:best_candidate}
  The \textbf{best candidate} is the node currently known to have the highest potential among all
  observed nodes. This notion is based on locally available information and may differ across
  nodes due to message loss or delays.
\end{definition}

\begin{definition}[Heartbeat]\label{def:heartbeat}
  A \textbf{heartbeat} is a periodically refreshed counter used to infer the continued activity
  of a node, particularly the current leader. If the counter is not refreshed within a given time
  window, the corresponding node is assumed to be unreachable or inactive for the purposes of
  leader election.
  Heartbeats are consumed by local tasks and may expire independently for the leader and the best
  candidate.
\end{definition}

\begin{definition}[Leader Check]\label{def:leader_check}
  A node declares itself leader if the following conditions hold:
  \begin{itemize}
    \item the node has won the most recent election round;
    \item the node's potential is greater than or equal to the potential of the current best
      candidate.
  \end{itemize}
\end{definition}

\begin{definition}[Election Time Scale Factor]\label{def:time_scale}
  The election time scale factor is a local, positive integer associated with each node.
  It multiplicatively scales the execution frequency of the Potential Election task.
  The factor has a minimum value of 1.

  The factor is \textbf
  {
    increased when a node initiates an election while not being the best candidate,
    decreased upon reception of valid leader heartbeats
  }, and \textbf
  {
    reset to its minimum value when both leader and best-candidate heartbeats expire
  }, enabling fast recovery after silence.
\end{definition}

\section{Ideal Solution}

The leader election problem defined in Section~\ref{prob:leader_election} is well studied in
the context of distributed systems. A widely adopted solution is the \textbf{Raft} consensus
algorithm~\cite{4}, which provides strong guarantees of safety and liveness by electing a leader
as part of a replicated state machine protocol.

Raft assumes a partially synchronous system, reliable point-to-point communication channels,
and stable node membership. Leader election in Raft is tightly coupled with log replication and
quorum-based agreement, resulting in well-defined behavior even in the presence of failures.
From a theoretical standpoint, Raft represents an ideal solution for achieving consistent
leadership and consensus in distributed systems.

However, these guarantees come at the cost of significant protocol complexity and communication
overhead. Raft relies on frequent heartbeat exchanges, persistent state, and multiple rounds of
message exchanges per election. Moreover, it assumes the availability of reliable,
connection-oriented communication channels between nodes.

In the context considered in this work—characterized by unreliable broadcast communication,
high message loss, lack of direct addressing, and severely constrained embedded
platforms—the assumptions required by Raft do not hold. As a result, while Raft can be considered
an ideal reference solution in terms of correctness guarantees, it is impractical for the target
environment addressed by this protocol.


% \section{Ideal Solution}
% This problem is not new and there already exists a good solution which is \textbf{RAFT}\cite{4}
% but it has some problem for my use case. First of all it's really an overkill, second it can be
% pretty heavy on the number of message sent.

\section{Actual Solution}

The solution implemented in \textbf{NearestAP} is inspired by the leader election mechanism of
\textbf{RAFT}\cite{4}, but is adapted to operate over unreliable broadcast media without any
connection-oriented guarantees.

Each node maintains a local view of the network, including the current leader and the
\textbf{best candidate}(see Definition~\ref{def:best_candidate}). Nodes infer leadership
by observing the continuous dissemination of identifiers, potentials, and election-related messages,
rather than relying on direct acknowledgments.

To prevent persistent election conflicts and excessive message traffic, NearestAP introduces an
adaptive election timing mechanism. Election attempts are not performed at a fixed rate; instead,
each node dynamically adjusts the frequency of election initiation based on observed leader
stability, candidate dominance, and network silence. Nodes that are unlikely to win an election
progressively suppress their own election activity, while the most promising candidate retains a
higher election rate.

Adaptive timing is implemented through a local election time scale factor.
Nodes that repeatedly attempt elections without being the strongest known candidate progressively
reduce their election frequency, allowing dominant candidates to emerge without continuous
contention. Conversely, prolonged silence from both the leader and the best candidate causes
nodes to restore aggressive election behavior, ensuring recovery from message loss or network
partitions.

In the absence of both leader and best-candidate signals, nodes autonomously recover by resetting
their election timing, ensuring continued progress even under severe message loss.


\section{Protocol Design Overview}

This section provides a high-level overview of the NearestAP protocol architecture.
The protocol is structured around three main elements: internal node state, message types,
and scheduled tasks. Together, these components implement a decentralized leader election
mechanism over unreliable broadcast communication.

Each node maintains a local internal state containing information about its own potential,
the currently known leader, the best candidate observed so far, and the status of any ongoing
election. This state is updated both periodically and in response to incoming messages.

Communication between nodes occurs through a small set of message types, each carrying information
relevant to leader election, such as node identifiers, potentials, and election rounds. Messages
are exchanged using unreliable broadcast, and no assumptions are made regarding delivery
guarantees or ordering.

Protocol logic is executed through a set of periodic and reactive tasks. Periodic tasks are
responsible for broadcasting leader heartbeats and initiating elections when appropriate, while a
reactive task processes incoming messages and updates the internal state accordingly.
The interaction between these tasks allows nodes to infer leadership information and suppress
unnecessary elections.

The following subsections describe these components in detail, starting with the internal state
representation, followed by message formats and task behavior.


% \section{Design}


\subsection{Node Internal State}

Each node maintains a local internal state composed of several logically distinct components:

\begin{itemize}
  \item \textbf{Topology}: describes the assumed network configuration, including the total number
    of nodes.
    \footnote{
      In the current version of the protocol, the network topology is static and cannot change at
      runtime.
      Nodes may become temporarily unreachable due to message loss, but they are still considered
      members of the network.
    }

  \item \textbf{Node Potentials} (Definition~\ref{def:potential}): the node stores the potential
    values of:
    \begin{enumerate}
      \item the local node;
      \item the current leader;
      \item the current best candidate (Definition~\ref{def:best_candidate}).
    \end{enumerate}

  \item \textbf{Heartbeat Counters}: two counters used to infer node activity:
    \begin{enumerate}
      \item a leader heartbeat counter, refreshed upon reception of valid leader heartbeat
        messages;
      \item a best candidate heartbeat counter, refreshed upon reception of messages referring to
        the current best candidate.
    \end{enumerate}

  \item \textbf{Node Identifiers}:
    \begin{enumerate}
      \item the identifier of the current leader;
      \item the identifier of the current best candidate.
    \end{enumerate}

  \item \textbf{Vote Information}: election-related state, including:
    \begin{enumerate}
      \item the total number of nodes in the network;
      \item the number of positive votes received during the current election;
      \item the current election round;
      \item two boolean flags indicating:
        \begin{itemize}
          \item whether the node has already initiated an election in the current round;
          \item whether the node has already voted in the current round.
        \end{itemize}
    \end{enumerate}
\end{itemize}


% \subsection{Node Internal State}
% Each node has an internal state which is divided in 4 categories:
% \begin{itemize}
%   \item Topology: tells the node the current network configuration
%     \footnote
%     {
%       At the current state of the protocol the network configuration is static and cannot alter.
%       Nodes can be unreachable, but they will still be considered as a valid member of the 
%       network.
%     }
%   \item Node Potentials[\ref{def:potential}]: contain the Potentials of:
%     \begin{enumerate}
%       \item current node
%       \item leader\label{state:leader}
%       \item best candidate[\ref{def:best_candidate}]
%     \end{enumerate}
%   \item \textbf{Leader heartbit} \label{def:laeder_heartbit}
%   \item \textbf{Best candidate[\label{def:best_candidate}] heartbit}\label{def:best_candidate_heartbit}
%   \item \textbf{Leader} id: the current leader node
%   \item \textbf{Best candidate}[\label{def:best_candidate}] id: the current best candidate node
%   \item Vote Info: the info regarding the elections. It contains:
%     \begin{enumerate}
%       \item the total number of nodes in the network\label{state:node_network}
%       \item the number of positive vote that the node has received\label{state:consent}
%       \item the current round\label{state:round}
%       \item two flags which indicates:
%         \begin{itemize}
%           \item if the node already sent an election in the current election\label{state:election_sent}
%           \item if the node already voted in the round current election\label{state:voted}
%         \end{itemize}
%         \label{state:flags}
%     \end{enumerate}
% \end{itemize}


\subsection{Messages}

The protocol operates through a small set of message types exchanged using unreliable broadcast.
While the specific semantics of each message type are described in the following subsections,
all incoming messages are subject to a common set of preliminary checks.

Upon reception of any message, a node performs the following operations:

\begin{itemize}
  \item \textbf{Best candidate update}: the message is examined to determine whether it refers to
    a node that should be considered as the current best candidate
    (Definition~\ref{def:best_candidate}). In particular:
    \begin{enumerate}
      \item if the potential carried in the message is greater than the locally stored best
        candidate potential, the best candidate is updated accordingly;
      \item if the identifier carried in the message matches the current best candidate
        identifier, the corresponding heartbeat counter is refreshed.
    \end{enumerate}

  \item \textbf{Round consistency check}: if the message carries an election round number greater
    than or equal to the node's current round, the local round is updated if necessary.
    Messages referring to older rounds are ignored with respect to election state updates.
\end{itemize}

All message types include a logical \textbf{round} number $r \in \mathbb{N}$, which is used to
order election attempts and prevent obsolete information from influencing the current election
state.

In addition to updating logical state, the reception of heartbeats directly influences the
election timing mechanism by reducing the local election time scale factor.


% \subsection{Messages}
% The protocol work around the following message that each node can send.
% The following checks are executed for each message:
% \begin{itemize}
%   \item check if the message comes from the best candidate[\ref{def:best_candidate}]
%     in the network by checking if:
%     \begin{enumerate}
%       \item \textbf{potential}[\ref{def:potential}] is >= than
%         \textbf{current best candidate potential}[\ref{def:potential}].
%       \item id == current \textbf{best candidate}[\ref{def:best_candidate}] id.
%     \end{enumerate}
%     In both the cases the infos of the \textbf{best candidate} are updated and the 
%     \textbf{heartbit}[\ref{def:heartbit}] for the \textbf{best candidate}[\ref{def:best_candidate}] is increased.
%   \item Check if the \textbf{round} is greater or equal than its round.
%     If the message round < its round the heartbit[\ref{def:heartbit}] is discarded Else 
%     the node update its own \textbf{round.}
% \end{itemize}
% Each of them uses the \textbf{round} $\in \N$ to measure time.

\subsubsection{LeaderHeartbeat}\label{message:leader_heartbeat}

\textbf{LeaderHeartbeat}$(id, potential, round)$ is a message periodically broadcast by the
current leader to advertise its presence and propagate leadership information.
The message carries the following fields:

\begin{itemize}
  \item \textbf{id}: the identifier of the leader node;
  \item \textbf{potential}: the current potential of the leader;
  \item \textbf{round}: the election round associated with the leader’s current leadership.
\end{itemize}

Upon reception of a valid leader heartbeat, nodes update their local view of the leader state,
refresh the leader heartbeat counter, and suppress election attempts if the leader’s potential
remains competitive.


% \subsubsection{LeaderHeartbit}\label{message:leader_heartbit}
% LeaderHeartbit(id, potential, round): a message sent by the leader where:
% \begin{itemize}
%   \item id: leader id
%   \item potential: leader potential
%   \item round: when the message has been sent
% \end{itemize}

\subsubsection{NewElection}\label{message:new_election}

\textbf{NewElection}$(id, potential, round)$ is a message broadcast by a node that intends to
initiate a new leader election. The message carries the following fields:

\begin{itemize}
  \item \textbf{id}: the identifier of the candidate node initiating the election;
  \item \textbf{potential}: the current potential of the candidate;
  \item \textbf{round}: the election round associated with this election attempt.
\end{itemize}

Upon receiving a \textbf{NewElection} message, nodes compare the candidate’s potential with their
local state and decide whether to support the election or suppress it in favor of a stronger known
candidate.


% \subsubsection{NewElection}\label{message:new_election}
% NewElection(id, potential, round): a message to any node which wants to
% become leader where:
% \begin{itemize}
%   \item id: node id
%   \item potential: node potential
%   \item round: when the message has been sent
% \end{itemize}

\subsubsection{VoteResponse}\label{message:vote_response}

\textbf{VoteResponse}$(new\_leader, potential, round)$ is a message broadcast by a node to endorse
a candidate that has initiated a leader election. The message carries the following fields:

\begin{itemize}
  \item \textbf{new\_leader}: the identifier of the candidate node being endorsed;
  \item \textbf{potential}: the potential of the endorsed candidate;
  \item \textbf{round}: the election round associated with the corresponding \textbf{NewElection}
    message.
\end{itemize}

Vote response messages are sent only when a node decides to support a candidate. They serve both to
notify the candidate of received support and to propagate information about the endorsed node to
the rest of the network.


% \subsubsection{VoteResponse}\label{message:vote_response}
% VoteResponse(new\_leader, potential, round): a \textbf{positive} response to a
% \textbf{NewElection}[\ref{message:new_election}] where:
% \begin{itemize}
%   \item new\_leader: id of the node who sends the \textbf{NewElection}[\ref{message:new_election}]
%   \item potential: potential of the node who sends
%     the \textbf{NewElection}[\ref{message:new_election}]
%   \item round: round of the \textbf{NewElection}[\ref{message:new_election}]
% \end{itemize}


\subsection{Tasks}

The last major components of the protocol are the \textbf{tasks}, which define the local behavior
of each node. Tasks are executed independently on every node and are responsible for periodically
evaluating the node state, sending messages, and reacting to changes observed in the network.

Each task operates at a configurable execution frequency\footnote
{
  All task frequencies are configurable in the protocol implementation\cite{5}.
}, allowing the protocol to be tuned for different environments and hardware constraints.
Tasks do not communicate directly with each other; instead, they interact solely through the
node’s internal state and the exchange of protocol messages.

The following subsections describe the tasks that collectively implement leader monitoring,
message processing, and election initiation.


% \subsection{Tasks}
% The last major components of the protocol are the tasks that organize the actions and the
% responses of each node and each of them has its own fixed frequency\footnote
% {
%   all the frequency are configurable in the protocol implementation\cite{5} 
% }.

\subsubsection{Leader Alive}

The \textbf{Leader Alive} task is executed periodically by each node. Upon execution,
the node performs a \textbf{Leader check}[\ref{def:leader_check}] on its local state.
If the node determines itself to be the current leader, it broadcasts a
\textbf{LeaderHeartbeat}[\ref{message:leader_heartbeat}] message to all other nodes.

This task does not participate in leader election; its sole purpose is to advertise the continued
presence and validity of the current leader.

\subsubsection{Bus Reader}

The \textbf{Bus Reader} task is responsible for processing all incoming messages and updating the
local node state accordingly. For each received message, the node applies message-specific rules
based on the message type and its current internal state.

\paragraph{LeaderHeartbeat}
Upon receiving a \textbf{LeaderHeartbeat} message, the node evaluates the information carried by
the message to update its view of the leader and the best candidate:
\begin{itemize}
  \item If the heartbeat round is greater than the local round, the node updates its round and
    adopts the heartbeat sender as the current best candidate, updating the associated potential.
  \item If the heartbeat sender is already known as the best candidate, or if its potential is
    greater than or equal to the locally known best candidate potential, the node updates its
    best candidate information accordingly.
  \item If the heartbeat sender has a potential greater than or equal to both the local node
    potential and the current best candidate potential, the node adopts the heartbeat sender as
    the current leader and aborts any ongoing election by resetting election-related state
    (sent election flag, vote flag, and supporter count).
  \item If the heartbeat sender matches the current leader, the node refreshes the leader
    heartbeat counter and updates the stored leader potential.
\end{itemize}
Upon reception of a valid \textbf{LeaderHeartbeat}, the node reduces its election time scale
factor, progressively restoring the base election frequency. This mechanism accelerates
convergence toward a stable leader once leadership has been established.

\paragraph{NewElection}
Upon receiving a \textbf{NewElection} message, the node evaluates whether the proposing candidate
should be supported or suppressed. The node responds only if all the following conditions hold:
\begin{enumerate}
  \item the candidate potential is greater than the local node potential;
  \item the candidate potential is greater than or equal to the locally known best candidate
    potential.
\end{enumerate}
Notably, this rule also applies to the current leader, allowing a leader to voluntarily endorse a
stronger candidate without requiring leader failure or timeout.


If these conditions are satisfied:
\begin{itemize}
  \item the node abandons any ongoing election attempt by resetting its election state;
  \item the candidate is adopted as the new best candidate and its potential is stored;
  \item a \textbf{VoteResponse} message endorsing the candidate is broadcast.
\end{itemize}

During a single election round, a node may emit multiple vote responses, but only when endorsing
candidates with strictly increasing potential\footnote
{
  This allows nodes to revise their support as better candidates are discovered.
}.

\paragraph{VoteResponse}
Upon receiving a \textbf{VoteResponse} message, a node increments its local supporter
count only if:
\begin{enumerate}
  \item it has previously initiated an election;
  \item the response round matches the local election round;
  \item the endorsed candidate identifier matches the local node identifier.
\end{enumerate}

If the number of collected endorsements exceeds half of the total number of nodes in the network,
the node declares itself leader and updates its internal state accordingly by advancing the round,
resetting election-related flags, and clearing the supporter count.

If neither leader nor best-candidate heartbeats are observed for a sufficient period, the node
resets its election time scale factor, restoring full election aggressiveness and ensuring
self-stabilization.

\subsubsection{Potential Election}

The \textbf{Potential Election} task is responsible for periodically recomputing the local node
\textbf{potential}[\ref{def:potential}] and determining whether a new leader election should be
initiated.

The task is executed at a frequency derived from a user-defined base value, multiplied by a
node-local \textbf{election time scale factor}[\ref{def:time_scale}]. This factor allows each node
to dynamically adjust the aggressiveness of election attempts based on observed network conditions.

A node initiates a new election when at least one of the following conditions holds:
\begin{enumerate}
  \item the leader heartbeat[\ref{def:heartbeat}] has expired;
  \item the local node potential exceeds the stored leader potential by more than the configured
    tolerance[\ref{def:tolerance}], and no known candidate has a strictly higher potential.
\end{enumerate}

When an election is initiated, the node updates its internal election state as follows:
\begin{itemize}
  \item the number of positive votes is initialized to 1 (self-vote);
  \item the \textbf{election sent} flag is set to true;
  \item the \textbf{voted} flag is set to true;
  \item a \textbf{NewElection}[\ref{message:new_election}] message is broadcast, carrying the node
    identifier, current potential, and current round.
\end{itemize}

If the node initiating the election is \emph{not} the current best candidate[\ref{def:best_candidate}],
the election time scale factor is increased, thereby reducing the frequency of subsequent election
attempts. This mechanism suppresses redundant elections from nodes that are unlikely to win, reducing
network traffic under contention.

In addition, the following state updates are applied:
\begin{itemize}
  \item if the node is currently the leader, both the stored leader and local node potentials are
    updated to the newly computed value;
  \item if the node is the current best candidate[\ref{def:best_candidate}], the stored best
    candidate potential is updated accordingly;
  \item if the best candidate heartbeat[\ref{def:heartbeat}] has expired at the time of election
    initiation, the node adopts itself as the new best candidate and resets the election time scale
    factor to its minimum value.
  \item If neither leader nor best-candidate heartbeats are observed,
    the node resets the election time scale factor to its minimum value.
    This silence-driven reset ensures that election activity resumes aggressively after
    prolonged message loss or leader disappearance, preventing permanent suppression.
\end{itemize}



\input{bib}

\end{document}
